\documentclass{report}
\usepackage{enumitem}
\usepackage[table]{xcolor}
\usepackage[croatian]{babel}
\usepackage{longtable}
\usepackage{tikz}
\usepackage[most]{tcolorbox}
\usepackage{tikz-3dplot}
\usepackage{lscape}
\usepackage{xargs}
\usepackage{todonotes}
\usepackage{float}
\usepackage{subfigure}
\usepackage{multirow}
\usepackage{listings}
\usepackage{mathtools}
\usepackage{fontawesome}
\usepackage{amsthm}

\renewcommand{\thesection}{(\Roman{section}.) }

\title{Detekcija konvergencije izvora kod proračuna kritičnosti MC metodom}

\author{Fran Lubina}

\begin{document}

\maketitle

\chapter*{prijedlog}
Htio bih predloziti svoju mesh-free metriku za detekciju konvergencije.
Rad na seminaru bi podijelio u slijedece etape:
\begin{enumerate}[label=\arabic*)]
	\item Formalna definicija metrike
	\item Odabir statistickog testa
	\item Implementacija metrike i testa
	\item Validacija
	\item Pisani dio seminara
\end{enumerate}


\section{FORMALNA DEFINICIJA METRIKE}
zbroj K najvecih distanci za neki neutron.
Distanca je neka distanca iz skupa svih mogucih međusobnih distanci neutrona trenutne i prosle fisijske banke.

\begin{equation}
KGPD = \sum_{d in K} d
\end{equation}

\begin{equation}
\begin{split}
K = \{d(i, j) : |M(i, j)| >= N - K\}, \\
M(i. j) = \{d'(i', j') : d'(i', j') < d(i, j)\}, \\
i,j \in [0, N-1], n_i \in  S^{(n)}, n_j \in S^{(n-1)}
\end{split}
\end{equation}

\begin{longtable}{|p{90pt}|p{160pt}|p{160pt}|}
	\hline
	\textbf{izvor} & \textbf{ponašanje KGPD} & \textbf{ponašanje prostorne raspodjele} \\
	\hline
	\endhead % This repeats the header on each page
	kovergiran & mali nasumicni pomaci oko E[.] & male promjene, eigenmod\\ \hline
	nekovergiran & monoton rast & velike promjene\\ \hline
\end{longtable}

\section{STATISTIČKI TESTOVI}
TBD (Seminar 2)

\section{IMPLEMENTACIJA}
OPCIJE:
\begin{enumerate}[label=\arabic*)]
	\item moj kod
	\item modificirana verzija vaseg licenciranog MCNP koda (kako su to Ueki et al. učinili)
	\item modificirana verzija OpenMC-a
\end{enumerate}

Već sam implementirao 1.), ali nisam koliko brzo i tocno moj kod moze raditi sa primjerima za validaciju.
Javim za 1-2 tjedna, ako bude potrebe ima dovoljno vremena da se ostvari 2.) ili 3.)

\section{VALIDACIJA}
Nad problemima koje su koristili Ueki et al.
\begin{enumerate}
\item Whitesideov keff-of-the-world
\item Homogena kocka
\item Bazen sa nukl. gorivom
\end{enumerate}

Ueki et al nisu koristili nikakve formalne metode verifikacije
vec samo graficku usporedbu rezultat pomocu grafova i tablica.
Ja cu postupiti isto.

Potrebno je samo implementirati opisanu metriku, jer je Ueki vec
implementirao ostale u svom radu nad istim problemima.
Znaci usporedba se svodi na jednu tablicu i jedan graf.

\section{Teorijska razmatranja}
Koji $K/N$ je najbolji? (Seminar 2)



\pagebreak
\section*{Komentari}
Ovakva metrika je jos jednostavnija za implementaciju nego NCDS.
Vrijeme izvrsavanja je vjerovatno gore, ali to nije ni bitno jer elementi "petlje generacija"
nemaju znatan ucinak na ukupno vrijeme.

\end{document}

